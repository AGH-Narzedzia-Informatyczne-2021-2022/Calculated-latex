\pagebreak
\section{Karol Błaszczak}

\begin{center}
    \large{\textbf {Kinematyka i Formuła 1}} \\
\end{center}

\subsection{Wzory z kinematyki}
\begin{center}
    $s= \frac{1}{2} at^{2} $
    
    $a=  \frac{ \Delta  \upsilon }{ \Delta t} $
\end{center}

\subsection{Formuła 1}
    Zdjęcie \ref{fig:MV33}. przedstawia bolid Formuły 1 Maxa Verstappena 
\begin{figure}[h]
    \centering
    \includegraphics[width=0.50\textwidth]{MV33}
    \caption{Bolid RB16B Maxa Verstappena}
    \label{fig:MV33}
\end{figure}

\subsubsection{Kierowcy F1}
    Tabela \ref{tab:top5} zawiera listę najlepszych kierowców w sezonie 2021 wraz z ilością zdobytych przez nich punktów (stan na 05.11.2021 r.).
\begin{table}[h]
\centering
\begin{tabular}{rlll}
\hline
\multicolumn{4}{|c|}{\textbf{Top 5 kierowców w sezonie 2021}}                                                         \\ \hline
\multicolumn{1}{l}{Miejsce} & Kierowca           & Zespół                        & \multicolumn{1}{r}{Liczba punktów} \\
1.                          & Max Verstappen \textit{33}  & Red Bull Racing Honda         & 287,5                              \\
2.                          & Lewis Hamilton \textit{44}  & Mercedes-AMG Petronas F1 Team & 275,5                              \\
3.                          & Valtteri Bottas \textit{77} & Mercedes-AMG Petronas F1 Team & 185                                \\
4.                          & Sergio Perez \textit{11}    & Red Bull Racing Honda         & 150                                \\
5.                          & Lando Norris \textit{4}     & McLaren F1 Team               & 149                               
\end{tabular}
\caption{Oto tabela najlepszych kierowców na dzień 5 listopada 2021 r.}
\label{tab:top5}
\end{table}

W tym roku pojawiło się kilku nowych kierowców
\begin{itemize}
    \item Nikita Mazepin (Uralkali Haas F1 Team)
    \item Mick Schumacher (Uralkali Haas F1 Team)
    \item Yuki Tsunoda (Scuderia AlphaTauri Honda)
\end{itemize}

W klasyfikacji konstruktorów prowadzą kolejno:
\begin{enumerate}
    \item Mercedes-AMG Petronas F1 Team
    \item Red Bull Racing Honda
    \item McLaren F1 Team
\end{enumerate}

\subsubsection{Sezon F1 2021}
\begin{center}
    \underline{Krótki opis aktualnego sezonu}
\end{center}

\setlength{\parindent}{10ex}
    \textbf{Aktualny sezon F1 wydaje się być bardzo ciekawy, wszyscy wyczekują w napięciu kto zostanie nowym mistrzem świata a zostało przed nami jeszcze 5 wyścigów.}\par
    \underline{GP Meksyku, } \textit{GP Brazylii, } \textbf{GP Kataru, \textit{GP Arabii Saudyjskiej, } \underline{GP Abu Zabi.}}