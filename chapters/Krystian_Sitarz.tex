\pagebreak
\section{Krystian Sitarz}

\subsection{Twierdzenie Pitagorasa}
$a^2 + b^2 = c^2$
\begin{center}
Geometryczne odzwierciedlenie twierdzenia Pitagorasa \ref{fig:pitagoras}.
\end{center}
\begin{figure}[h]
    \centering
    \includegraphics[width=0.35\textwidth]{pitagoras}
    \caption{Twierdzenie}
    \label{fig:pitagoras}
\end{figure}

\subsection{Tabela}

\begin{center}
Tabela \ref{tab:pitagoras} obrazuje twierdzenie.
\end{center}

\begin{table}[htbp]
\centering
\begin{tabular}{||c c c||} 
 \hline 
 $a$ & $b$ & $c$  \\ [0.3ex] 
 \hline\hline
 3 & 4 & 5  \\ 
 \hline\hline
 $a^2$ & $b^2$ & $c^2$  \\
 \hline\hline
 9 & 16 & 25 \\ [0.11ex] 
 \hline
\end{tabular}
\caption{Tabela obrazująca twierdzenie.}
\label{tab:pitagoras}
\end{table}

\subsection{Wielcy matematycy}

\begin{itemize}
    \item Banach
    \item Arystoteles
\end{itemize}

\begin{enumerate}
    \item Archimedes
    \item Galileusz
  
\end{enumerate}

\subsection{O matematyce}

W matematyce nie ma drogi specjalnie dla  \underline{królów}. \par
\vspace {0.2cm}
Matematyka jest drzwiami i kluczem do nauki.\par

