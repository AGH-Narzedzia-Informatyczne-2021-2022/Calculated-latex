\section{Kacper Majchrzak} 
\label{sec:kacmaj}

Oto przykładowe zdjęcie macierzy diagonalnej (Figure~\ref{fig:macdiag})

\begin{figure}[htbp]
    \centering
    \includegraphics[width=0.25\textwidth]{pictures/macierzdiag.png} 
    \caption{Wyznacznik tej macierzy będzie równy 0}
    \label{fig:macdiag}
\end{figure}

\begin{center}
Tabela~\ref{tab:pot} pokazuje potęgi kolejnych liczb.  
\end{center}

\begin{table}[htbp]
\centering
\begin{tabular}{||c c c c c||} 
 \hline 
 $a$ & $a^0$ & $a^1$ & $a^2$ & $a^3$ \\ [0.3ex] 
 \hline\hline
 1 & 1 & 1 & 1 & 1 \\ 
 \hline
 2 & 1 & 2 & 4 & 8 \\
 \hline
 3 & 1 & 3 & 9 & 27 \\
 \hline
 4 & 1 & 4 & 16 & 64 \\
 \hline
 5 & 1 & 5 & 25 & 125 \\ [0.11ex] 
 \hline
\end{tabular}
\caption{Oto tabela pokazujaca potegowanie.}
\label{tab:pot}
\end{table}

Twierdzenie pitagorasa: $a^2+b^2=c^2$

Własności mnożenia macierzy:
\begin{itemize}
  \item Łączne
  \item Rozdzielne względem dodawania
  \item Istnieje element naturalny (macierz jednostkowa)
\end{itemize}

Obliczenie wyznacznika macierzy 2x2:
\begin{enumerate}
  \item Pomnóż wyraz $a_{11}$ z wyrazem $a_{22}$
  \item Pomnóż wyraz $a_{12}$ z wyrazem $a_{21}$
  \item Odejmij pierwszy iloczyn od drugiego: $a_{11}*a_{22}-a_{12}*a_{21}$
\end{enumerate}

Oto krótki tekst \textit{maąjcy na \emph{celu}} \textbf{zaznajmienie się} z \LaTeX, nie ma na celu niczego innego.

Drugi akapit \textit{}{ma pokazać \textbf{dokładnie} to samo.} 