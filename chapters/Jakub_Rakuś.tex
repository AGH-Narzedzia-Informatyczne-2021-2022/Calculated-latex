\section{Jakub Rakuś}

$E = mc^2$

To zdjecie przedstawia dynie \ref{fig:dynia}.

\begin{figure}[h]
    \centering
    \includegraphics[width=0.35\textwidth]{dynia}
    \caption{dynia}
    \label{fig:dynia}
\end{figure}

Ta tabela \ref{tab:numbers} zawiera liczby. Tabela \ref{tab:numbers} sklada sie z czterech kolumn.

\begin{table}[h]
\centering
\begin{tabular}{|l|lll|}
\hline
\textbf{Columns 1} & \multicolumn{1}{l|}{\textbf{Columns 2}} & \multicolumn{1}{l|}{\textbf{Columns 3}} & \textbf{Columns 4} \\ \hline
1.                 & \multicolumn{3}{c|}{1}                                                                                 \\ \hline
2.                 & \multicolumn{1}{l|}{2}                  & \multicolumn{1}{l|}{4}                  & 8                  \\ \hline
3.                 & \multicolumn{1}{l|}{3}                  & \multicolumn{1}{l|}{9}                  & 27                 \\ \hline
4.                 & \multicolumn{1}{l|}{4}                  & \multicolumn{1}{l|}{16}                 & 64                 \\ \hline
\end{tabular}
\caption{liczby}
\label{tab:numbers}
\end{table}

\begin{itemize}
    \item dodawanie
    \item odejmowanie
    \item mnożenie
    \item dzielenie
\end{itemize}

\begin{enumerate}
    \item DODAWANIE
    \item ODEJMOWANIE
    \item MNOŻENIE
    \item DZIELENIE
\end{enumerate}

\begin{center}
    \underline{Krótki tekst}
\end{center}


\setlength{\parindent}{10ex}
\textbf{Pierwszy akapit. \emph {Pierwszy akapit.}}\par
    \underline{Drugi} \textit{akapit.} \underline{Drugi} \textit{akapit.} \textbf{Drugi akapit. Drugi akapit. Drugi akapit. Drugi akapit.}
    
    
    


