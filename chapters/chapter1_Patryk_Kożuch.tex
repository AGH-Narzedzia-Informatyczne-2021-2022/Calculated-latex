\section{Patryk Kożuch}
\subsection{Ciąg Fibonacciego - wzór}

To ciąg liczb naturalnych określony rekurencyjnie w sposób \nohyphens{następujący}:

\begin{equation}
    F_{n} = \begin{cases}
                    1 & \text{Dla } n = 1 \text{ i } n = 2 \\
                    F_{n-2} + F_{n-1} & \text{Dla } n \geq 3 \\
                \end{cases}
\end{equation}

\subsection{Graficzna interpretacja}
\begin{figure}[h]
    \centering
    \includegraphics[width=\textwidth]{pictures/zlota_spirala.jpg}
    \caption{Graficzna interpretacja ciągu Fibonacciego. Krzywa poprowadzona przez przeciwległe wierzchołki kwadratów o bogach równych kolejnym wyrazom ciągu tworzy tak zwaną \emph{Złotą spiralę}}
    \label{fig:fibonacci_graficznie}
\end{figure}

\subsection{Kolejne wyrazy ciągu}
Poniższa tabela przedstawia wartości kolejnych wyrazów ciągu Fibonacciego oraz sposób ich obliczania.
\begin{table}[h]
\centering
\caption{Kolejne wyrazy ciągu Fibonacciego}
\label{tab:kolejne_wyrazy}
\begin{tabular}{|c|c|}
\hline
n & $F_{n}$ \\ \hline
1 & 1      \\ \hline
2 & 1      \\ \hline
3 & $F_{1} + F_{2} = 2$      \\ \hline
4 & $F_{2} + F_{3} = 3$     \\ \hline
5 & $F_{3} + F_{4} = 5$      \\ \hline
6 & $F_{4} + F_{5} = 8$      \\ \hline
\end{tabular}
\end{table}
\subsection{Znaczenie kombinatoryczne ciągu Fibonacciego}
\begin{enumerate}
    \item liczba ciągów o wyrazach równych 1 lub 2, które sumują się do liczby n, jest równa $F_{n+1}$
    \item liczba pokryć planszy $2 \times n$ kostkami domina $2 \times 1$ jest równa $F_{n+1}$
    \item liczba ciągów binarnych bez kolejnych jedynek (równoważnie zer) jest równa $F_{n+2}$
    \item liczba podzbiorów zbioru \{ 1, \dots, n \} bez kolejnych liczb jest równa $F_{n+2}$
    \item liczba ciągów binarnych bez nieparzystej długości ciągów jedynek jest równa $F_{n+1}$
    \item liczba ciągów binarnych bez parzystej długości ciągów zer lub jedynek jest równa 2 $F_n$.
\end{enumerate}

\subsection{Mniej znane właściwości ciągu Fibonacciego}
\begin{itemize}
    \item Z wyjątkiem jednocyfrowych liczb ciągu Fibonacciego zawsze cztery albo pięć następujących po sobie wyrazów ciągu ma tę samą liczbę cyfr w układzie dziesiętnym.
    \item Jedynymi liczbami w ciągu Fibonacciego, będącymi kwadratami liczb całkowitych są 0, 1 i 144.
    \item Co trzecia liczba Fibonacciego jest podzielna przez 2, co czwarta – przez 3. Ogólniej: jeśli numer $n$ dzieli się przez $k$, to liczba $F_{n}$ dzieli się przez $F_{k}$.
\end{itemize}
\subsection{Życieroys Leonardo Fibonacci'ego}
\setlength{\parindent}{30pt}
Jego ojciec, \textbf{Guglielmo} z rodziny \textit{Bonacci}, zajmował stanowisko dyplomatyczne w Afryce północnej i Fibonacci tam właśnie się kształcił. Pierwsze lekcje matematyki pobierał od arabskiego nauczyciela w mieście Boużia (dziś algierska Bidżaja). Dużo podróżował najpierw razem z ojcem, później samodzielnie, odwiedzając i kształcąc się w takich miejscach jak \textbf{Egipt, Syria, Prowansja, Grecja i Sycylia}. W czasie swych podróży po Europie i po krajach Wschodu miał okazję poznać osiągnięcia matematyków arabskich i hinduskich, między innymi \textbf{\textit{dziesiętny system liczbowy}}.

\underline{Około 1200} roku Fibonacci zakończył podróże i powrócił do Pizy. Tam właśnie opracował swój słynny ciąg, którego interpretacja graficzna znajduje się na rysunku nr \ref{fig:fibonacci_graficznie}. Kolejne wyrazy możemy zobaczyć w tabeli nr \ref{tab:kolejne_wyrazy}.