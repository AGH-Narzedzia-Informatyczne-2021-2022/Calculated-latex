\pagebreak
\section{Bartosz Pobudejski}

\begin{center}
    \large{\textbf {Wzory skróconego mnożenia i statki}} \\
\end{center}
\rule{\textwidth}{0.5pt}
\subsection{Wzory skróconego mnożenia}
\textbf UWAGA!
\begin{equation}
    (x+y)^2 \neq x^2+y^2\quad \land \quad x,y \in \mathbb {R}
\end{equation}
Prawidłowe wzory na wzory skróconego mnożenia:
\begin{gather*}
    (x+y)^2 = x^2+2\cdot x \cdot y + y^2\quad \land \quad x,y \in \mathbb {R} \\
    (x+y)^3 = x^3 +3\cdot x^2\cdot y + 3\cdot x\cdot y^2 + y^3\quad \land \quad x,y \in \mathbb {R}
\end{gather*}
\begin{figure}[h!]
    \centering
    \includegraphics[width=0.45\textwidth]{trojkatPascala.png}
    \caption{Trójkąt Pascala służy do wyznaczania kolejnych wykładników}
    \label{fig:najwiekszy statek na swiecie}
\end{figure}
W ogólności:
\begin{equation}
    (a+b)^n = \sum^{n}_{k=0} {n \choose k}a^n-k\cdot b^k
\end{equation}
Gdzie:
\begin{gather*}
    {n \choose k} = \frac{n!}{(n-k)!}\\
    \text{}\\
    \sum^{n}_{k=0} = \underbrace{{n \choose 0}+{n \choose 1}+\cdots+{n \choose n} }_{n}
\end{gather*}

\pagebreak

\subsection{Statki}
\begin{figure}[!h]
    \centering
    \begin{subfigure}[b]{0.45\textwidth}
        \centering
        \includegraphics[width=\textwidth]{seawise_giant.png}
        \caption{}
        \label{fig:okret}
    \end{subfigure}
    \hfill
    \begin{subfigure}[b]{0.45\textwidth}
        \centering
        \includegraphics[width=\textwidth]{RoyalClipper.png}
        \caption{}
        \label{fig:zaglowiec}
    \end{subfigure}
    \caption{Dwa obrazy największych okrętow na świeciey (a) Największy statek (b) Największy żaglowiec}
\end{figure}

\textbf{\textit{Są różne rodzaje statków}}
\begin{itemize}
    \item żaglowce
    \item Promy
    \item Kontenerowce
\end{itemize}

\textbf{\textit{Ranking pod względem długości}}
\begin{enumerate}
    \item Royal Clipper (na zdjeciu~\ref{fig:zaglowiec})
    \item Batillus class
    \item Esso Atlantic
\end{enumerate}

\subsection{Przypadkowy tekst}
Nierostrzygniony najpiękniejszym wedle uchem mknie stole Kościuszkowskie. Mnie najwymowniejsza opis ludu. Moskal żołnierza wrzask bronie \underline{zgromadzona} zamieszkać wcześniéj rozgoworów Prosto Szabli przedziwnie bratni. \par
\vspace {0.2cm}
Buja Nikt Czas Terajewicza wasz prze mimo. Wiwat taką Kopę ustaw obrok kiermaszu dobry swój tłuszczy krąg paryskich talerzów. Stąd Tylko obelg podzielić wzory skał zrzędził dwókonną panie utrzymanie chowa. \par
\pagebreak
To jest tabela z róznymi danymi różnych ludzi (Table~\ref{tab:ludzie})
\begin{table}[h!]
\begin{tabular}{|l|l|l|l|l|l|l|}
\hline
\textbf{Imię} & \textbf{Nazwisko} & \textbf{Wiek} & \textbf{Wzrost} & \textbf{Waga} & \textbf{Hobby} & \textbf{Edukacja} \\ \hline
Bartosz       & Pobudejski        & 19            & 187             & 81            & Żeglarstwo     & średnie           \\ \hline
Jan           & Nowak             & 26            & 170             & 77            & Piłka nożna    & wyższe            \\ \hline
Marcin        & Kowalski          & 17            & 190             & 70            & Koszykówka     & \textit{-brak-}   \\ \hline
Aleksander    & Krupa             & 67            & 179             & 90            & Ksiązki        & wyższe            \\ \hline
\end{tabular}
\caption{bardzo potrzebna tabelka}
\label{tab:ludzie}
\end{table}
